Seasonal influenza virus A/H3N2 is a major cause of illness and death globally.
While vaccination is the most effective way to prevent infection, rapid accumulation of mutations in the surface protein hemagglutinin (HA) allows viruses to escape previous adaptive immunity due to vaccination or natural infection.
This antigenic drift requires vaccines to be updated regularly.
Due to a nearly one-year lag between the selection of vaccine strains and deployment of the vaccine to the public, effective vaccine strain selection requires predictions about future influenza populations.
Historically, experts predicted successful future viruses through manual inspection of hemagglutination inhibition (HI) assays, an experimental measure of antigenic drift.
Modern strain selection is increasingly informed by modeling of influenza fitness using HA sequences to estimate antigenic drift, functional constraint, and recent population growth.
The models rely entirely on genotypic information without accounting for the gold standard phenotypic measures of antigenic drift provided by HI assays.
We developed a novel open source forecasting framework based on these original fitness models that explicitly integrates genotypic and phenotypic measures of antigenic drift as well as modern experimental measurements of functional constraint from deep mutational scanning (DMS) experiments.
In contrast with previous work, our models minimize the distance between observed and estimated future populations using the Earth Mover's Distance metric instead of minimizing errors between observed and estimated phylogenetic clade frequencies.
We show that this integrated approach to forecasting can accurately estimate the sequence compositions of simulated and natural populations.
These forecasts capture dynamics of influenza clades and enable the identification of optimal individual strains for vaccine composition.
Using this framework and a modern machine learning approach to model testing, we found that phenotypic measures of antigenic drift were more consistently predictive of future influenza populations than sequence-only measures while the opposite was true for measures of functional constraint.
Importantly, the combination of antigenic drift and mutational load was the most predictive of future success.
We provide real-time forecasts of seasonal influenza A/H3N2 populations based on our combined model of antigenic drift and mutational load as part of Nextstrain.
As our open source framework allows definition of fitness metrics per sample using tidy data frames and does not require phylogenetic clades for forecasts, we hope it paves the way for future forecasting efforts with pathogens that cannot be readily analyzed by standard phylogenetic methods including recombinant viruses, bacteria, and fungi.
