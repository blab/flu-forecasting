\documentclass[12pt]{article}

% graphicx package, useful for including eps and pdf graphics
\usepackage{graphicx}
\DeclareGraphicsExtensions{.png,.png,.jpg}

% basic packages
\usepackage{color}
\usepackage{parskip}
\usepackage{float}
\usepackage{microtype}
\usepackage{url}
\usepackage{hyperref}

% text layout
\usepackage{geometry}
\geometry{textwidth=17cm} % 15.25cm for single-space, 16.25cm for double-space
\geometry{textheight=22.5cm} % 22cm for single-space, 22.5cm for double-space

% helps to keep figures from being orphaned on a page by themselves
\renewcommand{\topfraction}{0.85}
\renewcommand{\textfraction}{0.1}

% bold the 'Figure #' in the caption and separate it with a period
% Captions will be left justified
\usepackage[labelfont=bf,labelsep=period,font=small]{caption}

% cite package, to clean up citations in the main text. Do not remove.
\usepackage{cite}

\usepackage{authblk}
\renewcommand\Authands{ \& }
\renewcommand\Authfont{\normalsize \bf}
\renewcommand\Affilfont{\small \normalfont}
\makeatletter
\renewcommand\AB@affilsepx{, \protect\Affilfont}
\makeatother

\title{Antigenic phenotypes and phylogenetic metrics improve forecasts of seasonal influenza A/H3N2 evolution}
%\title{Improved forecasts of seasonal influenza A/H3N2 evolution}
%\title{Integrative forecasting of seasonal influenza A/H3N2 by genotype and phenotype}
%\title{Experimentally informed forecasts of seasonal influenza A/H3N2}
%\title{Long-term forecasts of seasonal influenza reveal historical contingency of fitness metrics}

\author[1,2]{John Huddleston}
\author[2]{Richard A.\ Neher}
\author[1]{Trevor Bedford}

\affil[1]{Vaccine and Infectious Disease Division, Fred Hutchinson Cancer Research Center, Seattle, WA, USA}
\affil[2]{Moleculary and Cell Biology, University of Washington, Seattle, WA, USA}
\affil[3]{Biozentrum, University of Basel, Basel, Switzerland}

\date{}

\begin{document}

\begin{abstract}
Seasonal influenza virus A/H3N2 is a major cause of illness and death globally.
While vaccination is the most effective way to prevent infection, rapid accumulation of mutations in the surface protein hemagglutinin (HA) allows viruses to escape previous adaptive immunity due to vaccination or natural infection.
This antigenic drift requires vaccines to be updated regularly.
Due to a nearly one-year lag between the selection of vaccine strains and deployment of the vaccine to the public, effective vaccine strain selection requires predictions about future influenza populations.
Historically, experts predicted successful future viruses through manual inspection of hemagglutination inhibition (HI) assays, an experimental measure of antigenic drift.
Modern strain selection is increasingly informed by modeling of influenza fitness using HA sequences to estimate antigenic drift, functional constraint, and recent population growth.
The models rely entirely on genotypic information without accounting for the gold standard phenotypic measures of antigenic drift provided by HI assays.
We developed a novel open source forecasting framework based on these original fitness models that explicitly integrates genotypic and phenotypic measures of antigenic drift as well as modern experimental measurements of functional constraint from deep mutational scanning (DMS) experiments.
In contrast with previous work, our models minimize the distance between observed and estimated future populations using the Earth Mover's Distance metric instead of minimizing errors between observed and estimated phylogenetic clade frequencies.
We show that this integrated approach to forecasting can accurately estimate the sequence compositions of simulated and natural populations.
These forecasts capture dynamics of influenza clades and enable the identification of optimal individual strains for vaccine composition.
Using this framework and a modern machine learning approach to model testing, we found that phenotypic measures of antigenic drift were more consistently predictive of future influenza populations than sequence-only measures while the opposite was true for measures of functional constraint.
Importantly, the combination of antigenic drift and mutational load was the most predictive of future success.
We provide real-time forecasts of seasonal influenza A/H3N2 populations based on our combined model of antigenic drift and mutational load as part of Nextstrain.
As our open source framework allows definition of fitness metrics per sample using tidy data frames and does not require phylogenetic clades for forecasts, we hope it paves the way for future forecasting efforts with pathogens that cannot be readily analyzed by standard phylogenetic methods including recombinant viruses, bacteria, and fungi.

\end{abstract}

\maketitle


\section*{Introduction}

Seasonal influenza virus infects 5--15\% of the global population every year causing an estimated 250,000 to 500,000 deaths annually \cite{flufactsheet}.
Vaccination remains the most effective public health response available.
However, frequent viral mutation results in viruses that escape previously acquired human immunity.
The World Health Organization (WHO) selects vaccine viruses to match circulating viruses, but because the process of vaccine development and distribution requires several months to complete, accurate vaccine strain selection requires a prediction of which viruses will predominate approximately one year after vaccine viruses are selected.
Current vaccine predictions favor viruses that are distinct from prior vaccine viruses in the hemagglutinin (HA) protein, which acts as the primary target of human immunity.
The hemagglutination inhibition (HI) assay \cite{hirst1943studies} is used to measure the degree of cross-reactivity between pairs of circulating viruses.
HI assays are fundamental for vaccine strain selection, but they are laborious and low-throughput compared to genome sequencing \cite{Wood:2012ii}.
As a result, researchers have developed computational methods to predict influenza fitness from sequence data alone \cite{Luksza:2014hj,Steinbruck:2014kq,Neher:2014eu}.

Despite the promise of these sequence-only models, they explicitly omit experimental measurements of antigenic or functional phenotypes.
Recent developments in computational methods and influenza virology have made it feasible to integrate these important metrics of influenza fitness into a single predictive model.
For example, phenotypic measurements of antigenic drift are now accessible through phylogenetic models \cite{Neher:2016hy} and functional phenotypes for HA are available from deep mutational scanning experiments \cite{Lee2018}.
We describe an approach to integrate previously disparate sequence-only models of influenza evolution with high-quality experimental measurements of antigenic drift and functional constraint.

\textit{Brief summary of model implementation here.}

\section*{Results}

\subsection*{Models accurately forecast evolution of simulated populations of A/H3N2-like viruses}

To test our model framework, we simulated the long-term evolution of five independent populations of A/H3N2-like viruses for 5,000 generations each using SANTA-SIM \cite{Jariani2019}.
We seeded each simulated population with the full length HA from A/Beijing/32/1992 such that all simulated sequences contained the signal peptide, HA1, and HA2 domains.
We defined purifying selection across all three domains, allowing the preferred amino acid at each site to change at a fixed rate over time.
We additionally defined exposure-dependent selection for 129 putative epitope sites in HA1 \cite{Wolf:2006da} to impose an effect of cross-immunity that would allow mutations at those sites to increase viral fitness despite underlying purifying selection.
These selective constraints produce phylogenetic structures and accumulation of epitope and non-epitope mutations that are consistent with phylogenies of natural A/H3N2 HA (Supplemental Figure \ref{sup_fig:simulated_h3n2_ha_phylogeny}).

Next, we fit models to 20 years of simulated sequences corresponding to generations 1000--3000 where 100 generations represents one year.
In the absence of any experimental data for these simulated sequences, we only fit models with sequence-based fitness metrics including epitope cross-immunity, non-epitope mutations, LBI, and delta frequency.

\textit{Model results here for sequence-based metrics relative to naive model}

\textit{Repeat simulations with multiple epochs to show model coefficients can reflect these underlying changes to evolutionary constraints.}

\subsection*{Models based on X metrics most accurately forecast natural populations of A/H3N2 viruses}

\subsection*{Model accuracy and coefficients reflect historical patterns of A/H3N2 evolution}

\subsection*{Composite models outperform models with individual fitness metrics}

\subsection*{Models enable selection of vaccine candidate strains}

\subsection*{Forecasts predict the rise of 3c3.A and A1b clades in February 2020}

\section*{Discussion}

\section*{Methods}

\subsection*{Strain selection}

\subsection*{Multiple sequence alignment}

\subsection*{Phylogenetic interference}

\subsection*{Frequency estimation}

\subsection*{Fitness model}

\subsection*{Model fitting and evaluation}

\subsubsection*{Model targets}

\subsubsection*{Time-series cross-validation}

\subsection*{Fitness metrics}

\subsubsection*{Antigenic drift}

\subsubsection*{Functional constraint}

\subsubsection*{Clade growth}


\clearpage

\bibliographystyle{nih}
\bibliography{manuscript}

\clearpage

\setcounter{figure}{0}
\setcounter{table}{0}
\renewcommand{\thefigure}{S\arabic{figure}}
\renewcommand{\thetable}{S\Roman{table}}

\section*{Supplemental Material}

\begin{figure*}[h]
  \begin{center}
  \includegraphics[width=\textwidth]{figures/cross-validation-for-simulated-populations.png}
  \caption{
  Time-series cross-validation scheme for simulated populations.
  Models were trained in six-year sliding windows (grey lines) and validated on out-of-sample data from validation timepoints (filled circles).
  Validation results from 30 years of data were used to iteratively tune model hyperparameters.
  After fixing hyperparameters, model coefficients were fixed at the mean values across all training windows.
  Fixed coefficients were applied to 10 years of new out-of-sample test data (open circles) to estimate true forecast errors.
  }
  \label{sup_fig:cross_validation_for_simulated_populations}
  \end{center}
\end{figure*}

\begin{figure*}[h]
  \begin{center}
  \includegraphics[width=\textwidth]{figures/simulated-h3n2-ha-phylogeny.png}
  \caption{
  Phylogeny of A/H3N2-like HA sequences sampled between the 24th and 30th years of simulated evolution.
  The phylogenetic structure and rate of accumulated epitope and non-epitope mutations match patterns observed in phylogenies of natural sequences.
  Sample dates were annotated as the generation in the simulation divided by 200 and added to 2000, to acquire realistic date ranges that were compatible with our modeling machinery.
  }
  \label{sup_fig:simulated_h3n2_ha_phylogeny}
  \end{center}
\end{figure*}

\begin{figure*}[t]
  \begin{center}
  \includegraphics[width=\textwidth]{figures/distance-of-simulated-populations-between-timepoints.png}
  \caption{
  Average weighted distance per year between simulated populations of viruses under the ``naive'' model.
  \tbc{I'm not sure this figure is worth including.}
  }
  \label{sup_fig:distance_of_simulated_populations_between_timepoints}
  \end{center}
\end{figure*}

\begin{figure*}[t]
  \begin{center}
  \includegraphics[width=\textwidth]{figures/unadjusted-composite-model-accuracy-and-coefficients-for-simulated-populations.png}
  \caption{Composite model a) accuracy and b) coefficients for simulated populations of A/H3N2-like viruses.}
  \label{sup_fig:unadjusted_composite_model_accuracy_and_coefficients_for_simulated_populations}
  \end{center}
\end{figure*}

\begin{figure*}[t]
  \begin{center}
  \includegraphics[width=\textwidth]{figures/validation-of-best-model-for-simulated-populations.png}
  \caption{
  Validation of best model for simulated populations of A/H3N2-like viruses.
  a) The correlation of estimated and observed clade growth rates shows the model's ability to capture clade-level dynamics without explicitly optimizing for clade frequency targets.
  b) The rank of the estimated best strain based on its distance to the future in the best model shows how often the model makes a good choice when forced to select a single representative strain for the future population.
  }
  \label{sup_fig:validation_of_best_model_for_simulated_populations}
  \end{center}
\end{figure*}

\begin{figure*}[h]
  \begin{center}
  \includegraphics[width=\textwidth]{figures/cross-validation-for-natural-populations.png}
  \caption{
  Time-series cross-validation scheme for natural populations.
  Models were trained in six-year sliding windows (grey lines) and validated on out-of-sample data from validation timepoints (filled circles).
  Validation results from 25 years of data were used to iteratively tune model hyperparameters.
  After fixing hyperparameters, model coefficients were fixed at the mean values across all training windows.
  Fixed coefficients were applied to four years of new out-of-sample test data (open circles) to estimate true forecast errors.
  }
  \label{sup_fig:cross_validation_for_natural_populations}
  \end{center}
\end{figure*}

\begin{figure*}[t]
  \begin{center}
  \includegraphics[width=\textwidth]{figures/distance-of-natural-populations-between-timepoints.png}
  \caption{Average weighted distance per year between natural populations of viruses under the ``naive'' model.}
  \label{sup_fig:distance_of_natural_populations_between_timepoints}
  \end{center}
\end{figure*}

\begin{table*}[ht]
  \begin{center}
    \begin{tabular*}{0.85\textwidth}{lrrl}
\toprule
                             Model &    Coefficients & \makecell{Distance to \\ future (AAs)} & \makecell[l]{Model $>$ naive \\ (N=23)} \\
\midrule
           non-epitope mutations + &  -0.68 +/- 0.34 &                          5.44 +/- 1.80 &                               18 (78\%) \\
                   \hspace{3mm}LBI &   1.03 +/- 0.40 &                                        &                                         \\
             oracle cross-immunity &   1.06 +/- 0.23 &                          5.59 +/- 1.30 &                               19 (83\%) \\
                               LBI &   1.12 +/- 0.51 &                          5.68 +/- 1.91 &                               17 (74\%) \\
               HI cross-immunity + &   1.39 +/- 0.19 &                          5.77 +/- 1.53 &                               18 (78\%) \\
 \hspace{3mm}non-epitope mutations &  -1.02 +/- 0.45 &                                        &                                         \\
               HI cross-immunity + &   1.28 +/- 0.47 &                          5.88 +/- 1.63 &                               19 (83\%) \\
 \hspace{3mm}non-epitope mutations &  -1.03 +/- 0.53 &                                        &                                         \\
                   \hspace{3mm}LBI &   0.05 +/- 0.50 &                                        &                                         \\
                           HI tree &   0.79 +/- 0.20 &                          6.00 +/- 1.48 &                               15 (65\%) \\
                 HI cross-immunity &   1.16 +/- 0.50 &                          6.04 +/- 1.57 &                               17 (74\%) \\
                   delta frequency &   0.79 +/- 0.47 &                          6.13 +/- 1.71 &                               16 (70\%) \\
             non-epitope mutations &  -0.99 +/- 0.30 &                          6.14 +/- 1.37 &                               17 (74\%) \\
            Koel epitope mutations &   0.37 +/- 0.48 &                          6.21 +/- 1.27 &                               19 (83\%) \\
                       DMS entropy &  -0.28 +/- 0.23 &                          6.38 +/- 1.33 &                               14 (61\%) \\
                             naive &   0.00 +/- 0.00 &                          6.40 +/- 1.36 &                                 0 (0\%) \\
                   DMS non-epitope &  -0.02 +/- 0.13 &                          6.45 +/- 1.42 &                                7 (30\%) \\
          linear HI mut phenotypes &   0.25 +/- 0.21 &                          6.46 +/- 1.50 &                               11 (48\%) \\
             HI sub cross-immunity &   0.31 +/- 0.37 &                          6.54 +/- 1.44 &                               10 (43\%) \\
            Wolf epitope mutations &   0.47 +/- 0.44 &                          6.57 +/- 1.27 &                                5 (22\%) \\
                epitope ancestor + &   0.53 +/- 0.52 &                          6.60 +/- 1.34 &                               12 (52\%) \\
 \hspace{3mm}non-epitope mutations &  -0.77 +/- 0.32 &                                        &                                         \\
            DMS mutational effects &   1.25 +/- 0.84 &                          6.75 +/- 1.95 &                               11 (48\%) \\
                 epitope mutations &   0.72 +/- 0.63 &                          6.83 +/- 1.41 &                                5 (22\%) \\
                  epitope ancestor &   0.23 +/- 0.51 &                          6.89 +/- 1.39 &                                8 (35\%) \\
          epitope cross-immunity + &   0.91 +/- 1.05 &                          6.94 +/- 1.39 &                                6 (26\%) \\
 \hspace{3mm}non-epitope mutations &  -0.97 +/- 0.39 &                                        &                                         \\
            epitope cross-immunity &   0.69 +/- 0.84 &                          7.06 +/- 1.37 &                                6 (26\%) \\
\bottomrule
\end{tabular*}

    \caption{
      Model performance with natural A/H3N2 populations relative to the naive model for all models tested.
      Models use one or more fitness metrics to minimize the distance between the population of HA amino acid sequences at timepoint, $t$, and those at a timepoint one year in the future, $t + 1$.
      The naive model assumes the populations at time $t$ and $t + 1$ are identical, effectively measuring the observed distance between the two timepoints.
      Better models produce estimates that are closer to the future population than the naive model.
    }
    \label{sup_table:complete_natural_model_selection}
  \end{center}
\end{table*}

\begin{table*}[ht]
  \begin{center}
    \input{tables/mutations_by_trunk_status.tex}
    \caption{
    Number of epitope and non-epitope mutations per branch by trunk or side branch status.
    Epitope sites were defined previously described \cite{Luksza:2014hj}.
    Annotation of trunk and side branch was performed as previously described \cite{Bedford:2015fj}.
    Mutations were calculated for the full validation tree for natural sequences samples between 1990 and 2015.
    }
    \label{sup_table:mutations_by_trunk_status}
  \end{center}
\end{table*}

\begin{figure*}[t]
  \begin{center}
  \includegraphics[width=\textwidth]{figures/unadjusted-DMS-model-accuracy-and-coefficients-for-natural-populations.png}
  \caption{
  Primary and alternate fitness metrics based on deep mutational scanning (DMS) preferences for seasonal influenza A/H3N2 strain A/Perth/16/2009.
  Metrics are increasingly less dependent on the sequence content of the DMS strain's genetic background from top to bottom.
  The most structurally naive metric of non-epitope mutations outperforms all of the DMS metrics, indicating that these metrics still overfit to the DMS strain's genetic background.
  }
  \label{sup_fig:unadjusted_DMS_model_accuracy_and_coefficients_for_natural_populations}
  \end{center}
\end{figure*}


\end{document}
